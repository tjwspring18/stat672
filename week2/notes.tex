\documentclass{article}
\usepackage{geometry}
\geometry{margin=1in}
\usepackage{amssymb}
\usepackage{amsmath}
\DeclareMathOperator*{\argmax}{argmax}
\pagenumbering{gobble}
\begin{document}
High-dimensional data exhibit \textbf{concentration of measure} and
\textbf{curse of dimensionality}.

\medskip

\textbf{Concentration of measure}: for high-dimensional objects, most of the volume is concentrated near the
surface. Consider any object $A$ in $\mathbb{R}^d$. Shrink $A$ by some small
amount $\epsilon$ to produce a new object $(1-\epsilon)A$. It is the case that
$Vol((1-\epsilon)A) = (1-\epsilon)^d Vol(A)$. It also is the case that for any
$x$, $1-x \leq e^{-x}$. So, $\frac{Vol((1-\epsilon)A)}{Vol(A)} = (1-\epsilon)^d
\leq e^{-\epsilon d}$. As $d \to \infty$, then $(1-\epsilon)^d \to 0$. This
implies that nearly all the volume of $A$ is contained in the portion of A not
included in $(1-\epsilon)A$.


\medskip

\textbf{Curse of dimensionality}: because of concentration of measure, some nice
things we would otherwise like to do become intractable. As dimensionality grows
and concentration of measure takes over, two random elements are likely to be
very far apart. This means that nearest neighbor methods don't work well.

\medskip

\textbf{Kernel density estimation (KDE)}: As motivation, suppose that we have some $d$-dimensional data
$\mathbf{X}$ and class labels $Y=k_1, k_2...$, and want to predict $Y$ based on
$\mathbf{X}$. In other words, we're considering $P(Y=k|X=x)$, which we know by
Bayes is equal to $\frac{P(X=x|Y=k)P(Y=k)}{P(X=x)}$. We don't particularly worry
about the bottom part of that equation since it doesn't depend on $Y$ and
becaues we often have decent information available for it. We often
have good information for the prior $P(Y=k)$ or can just use the sample
proportions. So the chief challenge is estimating the the density function
$f_k(x)$ associated with $P(X=x|Y=k)$. We have two ways of doing this:
parametric (assuming some model, e.g. Normal, and estimating parameters for it
via MLE) or non-parametric (make fewer assumptions).

KDE is a non-parametric method for $\hat{f}(x)$. As the simplest possible
example, we could use a histogram, where we divide the range of observed $x$'s
into some number $h$ of bins; count the proportion $\hat{p}_i$ of observations
in bin $i$ and divide by the length of the bin $h$. In this case, $\hat{f}_h(x)
= \sum_i^{m}\frac{\hat{p}}{h}$
(the subscript is to emphasize the degree to which this estimator depends on
choice of $h$). But, this is not a great estimator because observations very
close to the border only contribute to the density estimate for one bin, when it
seems like they should contribute to both. So, we use some function to weight
how much an observation should contribute to estimation of density at some
point, depending on how far away the observation is from the point (e.g.,
observation $x=100$ should not much increase our estimated density at $x=1$, but
should contribute a lot to our estimation of density at $x=101$). This weighting
function is called a \textit{kernel}. Implemented in one dimension, we have: $\hat{f}_{kde}(x) =
\frac{1}{n}\sum_{i=1}^{n}\frac{1}{h}K(\frac{X_i - x}{h})$. In multiple
dimensions, we have: $\hat{f}_{kde}(\mathbf{x}) =  \frac{1}{n}\sum_{i=1}^n
\frac{1}{|H|^{1/2}}K(\frac{\mathbf{X_i}-\mathbf{x}}{h})$

Our main requirements are that the kernel integrate to one, and that it be
symmetric. The most common are Epanechnikov and normal (Gaussian). Our big
choice is again the size of window to consider (e.g., the standard deviation in
our Gaussian kernel). A large window means high bias and low variance
(under-fit). A small window means low bias and high variance (over-fit). A
method of evaluating candidate kernels is the \textbf{mean integrated squared
error (MISE)}, aka \textbf{L2 risk function}, which is: $E(\hat{f} - f)^2 =
\int_{-\infty}^\infty (\hat{f}(x) -f(x))^2$. The optimal kernel might be one
minimizing this loss function. 

The problem is that as $d$ grows, the performance of KDE suffers, and there is
no estimator achieving better performance. The fundamental issue is that the CoD
means points become sparsely distributed in space and so $h$ has to become very
large to ensure that enough points fall into each neighborhood. 

\medskip

\textbf{Naive Bayes}: as a motivating example of why KDE is useful, however, we
consider Naive Bayes. We make a big assumption that given a particular label
$Y=k$, the $p$ features $\mathbf{X}_k$ are independent of each other, implying that 
the joint density of $\mathbf{X}_k$ is just the product of the individual
densities: $f(\mathbf{X}_k)= \prod_{j=1}^p f(\mathbf{X}_{jk})$. The nice thing
is that we can estimate these individual densities via KDE (since we're doing
them one at a time the CoD doesn't apply). So, we can end up with
$P(Y_i=k|\mathbf{X}_i=\mathbf{x}) =  P(Y_i=k)P(\mathbf{X}_i=\mathbf{x}|Y_i=k) =
\hat{p}_Y(k) \hat{f}_{X|Y}(\mathbf{x}|k)$, where $\hat{p}_Y(k)$ is our prior,
$\hat{f}_{\mathbf{X}|Y}(\mathbf{x}|k)$
is our KDE-based estimate of the likelihood of vector $\mathbf{x}$, and we don't
care about the constant denominator that does not depend on $Y$. We use the
\textbf{maximum a posteriori (MAP)} decision rule, which means we chose the
$\hat{y}$ that maximizes the aforementioned function: 
$$\hat{y} = \argmax_k \hat{p}_Y(k) \hat{f}_{X|Y}(\mathbf{x}|k)$$

This is a useful technique when the dimensionality is very high (e.g. NLP, spam
detection, etc.) in part because of the ease of computing KDEs.
\end{document}

